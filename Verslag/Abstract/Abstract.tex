\section*{Abstract}
An $N$-bodies model is built to simulate the formation of planets in a solar system. The model consists of a star and $N$ celestial bodies that orbit around the star according to Newton's equation, which is integrated numerically with LeapFrog. When collision happens, the two collided bodies will fuse into one body with a mass equals to the sum of the two masses. The position and velocity of this body at the following time steps will first be determined with the formula of conservation of momentum and then again by Newton's equation.\\
\\
To accelerate the simulations with large $N$, we implemented the Barnes-Hut Algorithm to reduce the number of calculations regarding Newton's equation and collision. The main questions we are interested are
\begin{itemize}
\item How does the number of planets vary during a 5000 years evolutions of a solar sytem?

\item How does the number of planets after 5000 years depends on $N$?
\end{itemize}
\leavevmode
\\
To answer these questions, we ran 5 simulations for each $N$, with $N$ ranging from $100$ to $1000$ with $\Delta N=100$, and determined the number of planets after each $10$ years during the simulations. We observed that, when $N$ is large enough, in our case for $N\geq 300$, the number of planets after 5000 years is around $8$ or $9$ and is independent of $N$. In the case of large $N$, only heavier and bigger planets are obtained.\\
\\
We have also investigated on the accuracy of our model. The analysis was done for $N=2$ and we observed that for planets that are more than $1$ AU to the sun, it's orbit is elliptical and the total energy of the system is conserved. This indicates that our model is accurate at certain level. However, with Richardson Extrapolation, the order of LeapFrog is estimated to be $1$ instead of $2$, which is in contradiction to the fact that LeapFrog is a well-known second order method. The reason for that is still unclear.