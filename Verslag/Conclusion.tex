\section{Conclusion}

During our research, we have found some really interesting results, including the mathematical tricks we used in our model.
First of all, we chose the Leapfrog method to integrate the forming of star systems.
This method has especially been developed for mechanical systems.
Moreover, the Leapfrog method is stable for oscillations and thus for uniform circular motion, what we have in our star system.\\

However, our model was far too slow for large numbers of initial celestial bodies, because we needed the distance between every two bodies.
To avoid this problem, we implemented the Barnes-Hut algorithm, which makes an estimation of the forces by grouping the bodies.
Happily, this algortithm reduced the running time from $\mathcal{O}(N^2)$ to $\mathcal{O}(N\log(N))$.\\

Our research is mainly concentrated on the influence of the number of initial celestial bodies.
We investigated how this influenced the number of planets during the 20000 years of formation of the star system and several properties of the formed planets like mass and radius.
We found that if $N\leq 1500$, a larger $N$ will result in a larger amount of planets after 20000 years and more specifically, also a larger number of gaseous planets. In contrast to this, the average of all the masses of the planets is independent of $N$. However, we believe that if $N$ is large enough, the number of planets will reach to a constant that is independent of $N$. \\

In the end, we investigated the accuracy of our model.
Here, we found that the energy is conserved well for planets with a distance of at least $1$ AU from the star with a time step of one month.
Therefore, we can conclude that our model is accurate enough, since we chose a time step of one month and all initial celestial bodies have distance to the star greater than $1$ AU. However, Richardson Extrapolation estimates an order $1$ for Leapfrog instead of $2$, which is a contradiction that we do not understand.





