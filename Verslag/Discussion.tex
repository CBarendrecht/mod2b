\section{Discussion}
Looking back at our working process, we think we made some good choices.
For instance, we used the Barnes-Hut algorithm succesfully and the choice for Leapfrog integration was good due to its stability for oscillatory movements.\\

However, the simulations still took quite a long running time, even though we had implemented the Barnes-Hut algorithm. It can be considered to use other programming language than Matlab like Python or C, which might work faster.\\

In our 1200 celestial bodies simulation, the system was empty after 19000 years, which is rather unrealistic. A possible explanation is that we might not have chosen a correct Barnes-Hut parameter, causing the model to be unaccurate. We did however confirm that our integration method Leapfrog did not cause inaccuracy of this level.\\

For further research, it might also be interesting to look at the influence of size of the solar system (max\(R\)) on the number of planets. In our case we choose max\(R\) to be $7$AU, but in reality, a solar system like ours is at least 40 AU. We expect that if we increase maxR, then there will also be less collisions. So a larger $N$ is needed to obtain a reasonable result. Also, for a larger maxR, the distribution of gaseous planets will have more influence on the system. Moreover, one can have a closer look at this distribution and improve it so it is more realistic.\\

Another option is expanding the system to a three-dimensional system.
One of the problems in this expansion is the direction of the initial velocities of the bodies.



