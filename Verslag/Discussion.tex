\section{Discussion}
Looking back at our working process, we think we made some good choices.
For instance, we used the Barnes-Hut algorithm succesfully and the choice for Leapfrog integration was good due to its stability for oscillatory movements.\\

In our 1200 celestial bodies simulation, the system was empty after 19000 years, which is rather unrealistic. A possible explanation is that we might not have chosen a correct Barnes-Hut parameter, causing the model to be unaccurate. We did however confirm that our integration method Leapfrog did not cause inaccuracy of this level.\\

The simulations had a really long running time, even though we had implemented the Barnes-Hut algorithm. This reduces the opportunity to validate the results.
Furthermore, we had to run all the simulations twice, since we noticed a small mistake in our code after the first set of simulations. 
Thus we need to check our code more often and maybe need to involve an algorithm that reduces the running time even more than Barnes-Hut.
Another option is avoiding the problem by using a faster computer.\\

For further research, it might be interesting to look at the influence of initial range where the bodies are placed in the beginning of the simulation, because this will involve the distribution of gaseous planets even more.
Moreover, one can have a closer look at this distribution and make it more realistic.
Another option is expanding the system to a three-dimensional system.
One of the problems in this expansion is the choice of the direction of the initial velocities of the bodies.



