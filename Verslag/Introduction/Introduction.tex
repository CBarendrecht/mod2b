\section{Introduction}
The formation of the Solar System is described by the Nebular Hypothesis. It is suggested that the Sun was born from the gravitational collapse of a giant molecular cloud called the Solar Nebula. 
The rotation of this cloud caused the cloud to flatten out and take the form of a disk. In a later stadium, a protoplanetery disk was formed, which later evolved into our planetary system.
During this evolution, planets are formed through a process known as accretion.
In this process, cosmic dust accumulates to become small planetesimals, these planetesimals then collide and fuse into a larger celestial body, which will eventually result in the formation of a planet.\\

Based on this idea, we first built a basic model in which \(N\) celestial bodies of mass ranging from $0.005\cdot M_{\oplus}$ to $0.01\cdot M_{\oplus}$ (where $M_{\oplus}=1$ Earth Mass) rotate around a star with mass $3.33\cdot 10^5 M_{\oplus}$ (roughly the mass of the Sun).
The mass of every celestial body is composed of a solid core and a gaseous atmosphere.
We set the center of mass of the entire system (planets and sun) at the origin of our coordinate system. 
The initial positions and velocities of all celestial bodies (with excepetion of the Sun) are determined at random, under the condition that the initial distance between the center of mass ranges between 1 AU and  maxR (1 AU is the distance between the Earth and the Sun). 
The initial position and velocity of the Sun are then determined such that the momentum of the center of mass is zero. 
The positions and velocities of all bodies are then numerically integrated by applying the Leapfrog scheme to the Newton's equation:
\begin{align}
\frac{d^2\vec{r}_i}{dt}=G\sum_{j\neq i}m_j\frac{\vec{r_j}-\vec{r_i}}{|\vec{r_j}-\vec{r_i}|^3}
\end{align}

We consider an isolated system, therefore there are no external forces working on these planets. 
The momentum of the center of mass is thus conserved and its position will therefore always be zero, which makes it a good choice for the center of our coordinate system. 
In our basic model, we simulated this solar system over 1000 years, with a time step (\(dt\)) of $1$ month.\\

It is certainly possible that two bodies will collide at a certain time. 
At that moment, Newton's equation can not be applied since we will divide by zero. 
In the case of a collision, we use the formula of elastic collision to update the positions and velocities for one step. 
To simulate accretion, the mass of the smaller body is then set to zero and the mass of the larger body is set as the sum of the two masses. 
The positions and velocities of the remaining body are then further determined by Newton's equation.\\

In our model, we simulated the evolution of a solar system over $5000$ years, with a time step of $1$ month. 
In order to formulate our research questions, we made the following definitions:\\
\begin{enumerate}
\item \textbf{maxR}: The size of a solar system in our model is characterized by $\text{max}R$, which is the radius of a solar system. 
\item \textbf{Escaping the solar system}: A celestial body has escaped our solar system if it's distance to the center of mass is larger than $2\cdot\text{max}R$ at a certain time.
\item \textbf{Planet}: A celestial body in our model is considered to be a planet if it did not escape the solar system and if it's mass is at least $0.06\cdot M_{\oplus}$ (mass of Mercury) and lower than $100\times 318M_{\oplus}$ (100 times Jupiter's mass).
\end{enumerate}
We then studied the evolution of this solar system by investigating the following main questions:
\begin{enumerate}
	\item  How many planets are there in total after $5000$ years? 
	How does the number of planets vary during this $5000$ years of evolution?

	\item What is the type, mass, radius, orbital period around the sun and the average distance of each planet to the sun after $5000$ years?

	\item How many bodies has escaped the solar system after $5000$ years?

\end{enumerate}
 