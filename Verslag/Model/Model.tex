\section{The Model}
As mentioned in the introduction, we will be simulating an \(N\)-Body problem including a heavy  object functioning as a star placed in the center of the system. 
The Model uses the LeapFrog integration method to approximate the new movements. 
Before we can apply Newton's equation however, we have to define our system of measurement and scale the corresponding physical constants.
Table \ref{tab:eenheden} displays the chosen Units for the corresponding physical quantity.

\begin{table}[h!]
\centering
\caption{The chosen Units of measurement}
\label{tab:eenheden}
\begin{tabular}{l|l|l}
  Physical quantity & Unit & SI-value \\ \hline
Time & 1 month & \(2.629\cdot 10^{6}\) seconds  \\ 
 Distance & 1 Astronomical Unit & \(1.496\cdot10^{11}\) meters   \\
 Mass & 1 Earth Mass & \(5.972\cdot10^{24}\) kilograms \\
\end{tabular}
\end{table}
Other units such as velocity, acceleration and density follow from these definitions. After these units are set, the gravitational constant is rescaled and the simpelest model can be created.
In this model, several parameters are relevant and can be chosen as desired. 
\begin{itemize}
	\item \textbf{Number of Celestial bodies}, denoted as \(p\), this parameter gives the number of celestial bodies present at the start of a simulation. This excludes the star, thus the system will initially contain a total of \(p+1\) objects.
	\item \textbf{Integration increment}, denoted as \(dt\), this parameter determines at which moments the system will integrate. In example, if \(dt=2\), new positions will be calculated every two months. By default we set \(dt=1\).
	\item \textbf{Integration time}, denoted as \(T\), this parameter indicates the duration of the simulation. If \(t=6\cdot10^4\), the evolution of the system will be observed over a time period of 5000 years.
	\item \textbf{Minimum Radius}, denoted as min\(R\), in the model, the celestial bodies start in a n annulus around the origin. This parameter sets the inner radius of this annulus and gives a lower limit to the distance between celestial bodies and the center of mass at \(t=0\).
	\item \textbf{Maximum Radius}, denoted as max\(R\), this parameter sets the outer radius of the annulus and gives an upper limit to the distance between a celestial body and the center of mass at \(t=0\).
	\item \textbf{Minimum Mass}, denoted as min\(M\), this parameter sets the minimum mass that a celestial body can have at \(t=0\).
	\item \textbf{Maximum Mass}m denoted as max\(M\), this parameter sets the maximum mass that a celestial body can have at \(t=0\).
\end{itemize}
%With these parameters chosen, 
%\(p\) celestial bodies are generated, distributed uniformly in distance from the center of mass.
%with a distance from the center of mass determined at random through a uniform distribution. 
If we identify our system with a 2 Dimensional surface, we can choose a location for each of the \(p\) celestial bodies in polar co\"ordinates, using a uniform distribution to determine both the distance between the body and the center of mass and the angle of the celestial body. 
The mass of the celestial body is determined uniformly as well.
